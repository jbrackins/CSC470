% !TEX root = SystemTemplate.tex
\chapter{Design  and Implementation}
 This section describes the design details for the overall system as well as individual major components. As a user, you will have a good understanding of the implemetation details without having to look into the code. Here is the  


\begin{itemize}
  \item Ask if the user if the program needs to generate test cases
  \item If the user's answer is yes: 
	 \begin{itemize}
		\item Get the requirements for test cases
 	 	\item Generate {\tt.tst} files and corresponding {\tt .ans} files using {\tt golden.cpp}
	\end{itemize}
  \item (If no) get all the {\tt .tst} files and add them to a vector 
  \item Get all {\tt .cpp} files and compile them into executable files
  \item Create summary file
  \item For each {\tt .cpp} files in the directory: 
 	\begin{itemize}
 		\item Create {\tt .log} file for current student
		\item For all {\tt .tst} file in the vector:
		\begin{itemize}
			 \item Run student file using current test case
 			\item If pass the increment the number of passed tests, and output the score to student's {\tt .log} file
			 \item If fail, check if it is a critical test, if so, the student has failed, else output the score to the {\tt .log} file
		\end{itemize}
	\end{itemize}
  \item Check if the user wants more cases
  \item If yes, then restart from the beginning
  \item If no:
	\begin{itemize}
  		\item Output student's overall grade to summary file
  		\item Close current student's {\tt .log} file	
	\end{itemize}
  
\item Check if there are more {\tt .cpp} files to be processed
  \item If yes, then repeat the previous steps
  \item If no:
	\begin{itemize}
 		\item Close summary file
 		\item End test program
	\end{itemize}
\end{itemize}


\section{Traversing Subdirectories }

\subsection{Technologies  Used}
The dirent.h library is used for traversing subdirectories.

\subsection{Design Details}

\begin{lstlisting}

bool change_dir(string dir_name)
{
    string path;
    if(chdir(dir_name.c_str()) == 0) 
    {
        path = get_pathname();
        return true;
    }
    return false;
}

bool is_dir(string dir)
{
    struct stat file_info;
    stat(dir.c_str(), &file_info);
    if ( S_ISDIR(file_info.st_mode) ) 
        return true;
    else 
        return false;
}
\end{lstlisting}

\section{Running the Program Using Test Cases }

\subsection{Technologies  Used}
The software was designed in the Linux Environment provided to the group by the University.



\subsection{Design Details}


\begin{lstlisting}
int run_file(string cpp_file, string test_case) //case_num
{
    //create .out file name
    string case_out(case_name(test_case, "out"));

    //set up piping buffers
    string buffer1("");
    string buffer2(" &>/dev/null < ");
    string buffer3(" > ");

    // "try using | "
    //construct run command, then send to system
    //./<filename> &> /dev/null  < case_x.tst > case_x.out
    buffer1 += cpp_file + buffer2 + test_case + buffer3 + case_out;
    system(buffer1.c_str());

    //0 = Fail, 1 = Pass
    return result_compare(test_case);
}
\end{lstlisting}

\section{Generating Test Cases}

\subsection{Technologies  Used}

\subsection{Design Details}

