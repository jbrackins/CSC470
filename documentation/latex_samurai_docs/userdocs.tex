% !TEX root = SystemTemplate.tex

\chapter{User Documentation}

This section contains the user guide, user documentation.

%\newpage   %% 
%%  The user guide can be an external document which is included here if necessary ...
%%  a single source is the way to go.

\section{User Guide}

This product is to make it easy for you to grade a computational computer program written in C++ language. In order to benefit from this product, the test.cpp file and Makefile for the testing program must be in the same directory. Located in the same directory as the test.cpp file should be a folder containing a .cpp file to be tested. The directory and .cpp names should match. In the .cpp file’s directory, you should have:

	\begin{itemize} 
  		\item The{\tt .cpp} file to be tested
  		\item Test files (with {\tt .tst} extensions) 
  		\item Corresponding solution files (with {\tt .ans} extensions)
	\end{itemize} 

As a user, all you need to do is type {\tt grade <filename>.cpp} in the terminal. It is assumed that you are using this product on  {\tt Windows} or  {\tt Linux} operating system. The program will run through all the test cases and compare the results from the  {\tt .cpp} file with the answer in  {\tt .ans} file. If the answers were similar, the {\tt .cpp} file will get 100\% credit for that partiular case, else the percentage will be zero. The program will run and return you a {\tt .log} file containing the grade for a given {\tt .cpp} file.
   


